%section-start setup
\documentclass{beamer}
%section-start declare document information!
\title[SMC Generation]{Sequential Monte Carlo for Constrained Generation}
\subtitle{An Application of Generative Models}
\author{Hannah Nelson}
\institute{University of California, Santa Barbara}
\date{2025}
%section-end
%section-start load packages!
\usepackage{graphicx}
\usepackage[backend=biber]{biblatex}
\usepackage{xcolor}
\usepackage{soul} % use the package named soul by Melchior Franz for highlighting
%section-end
%section-start add bibliography
\addbibresource{refs.bib}
%section-end
%section-start define macros
%section-start shiftleft
% A simple command to shift stuff left. meant to shift itemizes left because beamer doesn't play well with left margins for some reason.
\newcommand\shiftleft[1]{
    \begin{columns}
        \begin{column}{0.1\textwidth}
        \end{column}
        \begin{column}{0.9\textwidth}
            #1
        \end{column}
    \end{columns}
}
%section-end
%section-end
%section-end
%section-start document
\begin{document}
%section-start titlepage
\begin{frame}
\titlepage
\end{frame}
%section-end
%section-start overview
\begin{frame}
\frametitle{Overview}
%TODO make the overview last.
    \begin{itemize}
        \item The Problem: Contstrained Generation
        \item What is Sequential Monte Carlo?
            \begin{itemize}
                \item Importance Resampling
                \item Repeated Importance Sampling
            \end{itemize}
        \item A Selection of Solutions
            \begin{itemize}
                \item Feynman Kac Diffusion Steering (FKDS)
                \item Sequentially Ordered SMC (SOSMC)
                \item LLM Probabilistic Programming (LLaMPPS)
                \item Probabilistic Planning with SMC
            \end{itemize}
    \end{itemize}
\end{frame}
%section-end
%section-start the problem
\begin{frame}
\frametitle{The Problem: Constrained Generation}
    Imagine we want to:
    \begin{itemize}
        \item Generate an image which fits a description.
        \item Fill in a blank in text.
        \item Make a tree which doesn't intersect objects.
        \item Find a sequence of actions which yeilds a reward.
    \end{itemize}
    We need to generate a \textbf{solution} which obeys a \textbf{constraint}
\end{frame}
%section-end
%section-start visualizations of the problem.
\begin{frame}
\frametitle{The Problem: Visualizations} % TODO
    \includegraphics[totalheight=3cm]{resources/infilling.png}
    \footnote{\textcite{lew2023sequentialmontecarlosteering}}
    \includegraphics[totalheight=3cm]{resources/tree_growth.png}
    \footnote{\textcite{10.1145/2766895}}
    \includegraphics[totalheight=3.5cm]{resources/text_to_image.png}
    \footnote{\textcite{singhal2025generalframeworkinferencetimescaling}}
\end{frame}
%section-end
%section-start the approaches
\begin{frame}{There are many approaches!}
    \begin{columns}[t] % [t] aligns content to the top
        % --- Column 1 ---
        \begin{column}{0.3\textwidth}
            \textbf{Optimization Based}
            \begin{itemize}
                \item Beam Search
                \item Expectation Maximization
                \item Simulated Annealing
                \item Greedy Setpwise Generation
            \end{itemize}
        \end{column}
        % --- Column 2 ---
        \begin{column}{0.3\textwidth}
            \textbf{Conditional Based}
            \begin{itemize}
                \item Importance Sample
                \item Monte Carlo Markov Chain
                \item Classifier Free Guidance
                \item
                    \hl{Sequential Monte Carlo}
            \end{itemize}
        \end{column}
        % --- Column 3 ---
        \begin{column}{0.3\textwidth}
            \textbf{Hueristic Based}
            \begin{itemize}
                \item Wave Function Collapse
                \item Context Engineering
                \item Hueristic EM
                \item Chain of Thought
            \end{itemize}
        \end{column}
    \end{columns}
\end{frame}
%section-end
%section-start we want smc
\begin{frame}
\frametitle{We Want SMC Because}
%TODO
\end{frame}
%section-end
%section-start importance resampling
\begin{frame}
\frametitle{A naive approach: Importance Resampling}
    \includegraphics[totalheight=8cm]{resources/importance_resampling.png}
\end{frame}
%section-end
%section-start smc
\begin{frame}
    \frametitle{The Key Improvement Sequential Monte Carlo (SMC)}
    \includegraphics[totalheight=8cm]{resources/smc_as_iterated_importance_sampling.png}
\end{frame}
%section-end
%section-start fkds
\begin{frame}
\frametitle{Feynman Kac Diffusion Steering}
    \shiftleft{
        \begin{itemize}
            \item[The Paper:]
                \citetitle{singhal2025generalframeworkinferencetimescaling}
            \item[The Problem:]
                Generate an image which fits a satisfies a classification.
            \item[The Difficulty:]
                Classifier models cannot give accurate scores on partially generated images.
        \end{itemize}
    }
\end{frame}
%section-end
%section-start
\begin{frame}
\frametitle{FKDS: Approach}
%TODO make solution slide for the FKDS
1. Identify the constraint a valid solution must solve:

2. Make solutions which weigh themselves using approximate rewards obeying constraint:

3. Select an appropriate approximate reward.
\end{frame}
%section-end
%section-start sosmc
\begin{frame}
    \frametitle{Stochastically Ordered Sequential Monte Carlo (SOSMC)}
    %TODO change out the statements to match SOSMC
    \shiftleft{
        \begin{itemize}
            \item[The Paper:]
                \citetitle{10.1145/2766895}
            \item[The Problem:]
                3d Generation uses a branching generation structure. SMC requires a flat generation order.
            \item[The Difficulty:]
                Naively flattening the structure gets poor results due to generation order.
        \end{itemize}
    }
%TODO
\end{frame}
%section-end
%TODO make solution slide for the SOSMC
%section-start LLaMPPL
\begin{frame}
    \frametitle{Language Model Probabalistic Programing (LLaMPPL)}
    %TODO change out the statements to match LLaMPPL
    \shiftleft{
        \begin{itemize}
            \item[The Paper:]
                \citetitle{lew2023sequentialmontecarlosteering}
            \item[The Problem:]
                We need our models to fill in blanks in sentences.
            \item[The Difficulty:]
                Our blanks have are of unknown length.
        \end{itemize}
    }
%TODO
\end{frame}
%section-end
%TODO further explain the difficulty LLaMPPL
%TODO make solution slide for the LLaMPPL
%section-start PP-SMC
\begin{frame}
\frametitle{Probabilistic Planning with SMC}
    %TODO change out the statements to match PP-SMC
    \shiftleft{
        \begin{itemize}
            \item[The Paper:]
                \citetitle{piche2018probabilistic}
            \item[Aim:]
                Control as inference; Good plans are probable plans with high reward,
            \item[Key Idea:]
                SMC can find probable plans with high reward.
        \end{itemize}
    }
%TODO
\end{frame}
%section-end
%TODO make a slide showing the results of PP-SMC
%section-start Citations
\begin{frame}
\frametitle{}
%TODO
    \cite{singhal2025generalframeworkinferencetimescaling}
    \cite{piche2018probabilistic}
    \cite{10.1145/2766895}
    \cite{lew2023sequentialmontecarlosteering}
    \printbibliography
\end{frame}
%section-end
%section-start
\begin{frame}
\frametitle{}
%TODO
\end{frame}
%section-end
\end{document}
%section-end
